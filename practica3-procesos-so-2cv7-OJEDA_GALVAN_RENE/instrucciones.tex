\section{Comunicación entre procesos: Tubería sin nombre}

	El alumno realizará la compilación, ejecución y análisis de un programa que ejemplifica el uso de tuberías con la función pipe.

\begin{itemize}

	\item Compile y ejecute el programa:

		\begin{enumerate}

			\item tuberia-sin-nombre.c

		\end{enumerate}

\end{itemize}

	Realice la documentación requerida para el programa.
	
#include<stdio.h>
#include<unistd.h>
#include<sys/types.h>
#include<string.h>
#define MAX 256

int main () {	
pid_t hijo;
int tuberia[2];
char mensaje[MAX];
if ( pipe (tuberia) == -1) {	/* Se crea la tuberia y se valida la creación de la tuberia*/
perror("Error al crear la tuberia sin nombre con pipe");		
return(-1);
}
else {
hijo = fork();			/*Se crea el proceso hijo*/
switch (hijo) {			/*Se valida la creación del proceso hijo*/
case -1 :				
perror("Error al crear el proceso hijo con fork");  */Muestra un mensaje de error si no se crea el proceso*/		
return (-1);			
break;			
case 0 :				
while( read (tuberia[0], mensaje, MAX) > 0 && strcmp (mensaje, "FIN\n") != 0 )
	/*Valida la condicion de salida del programa*/				
printf("\n El proceso hijo recibe el mensaje: %s \n", mensaje);		/*Imprime un mensaje*/		
close(tuberia[0]);		/*Cierra la tuberia del proceso Padre*/	
close(tuberia[1]);		/*Cierra la tuberia del proceso Hijo*/	
return (0);			
break;	
default :				
do{					
printf("\n Escriba el mensaje del padre: ");					
if ( fgets (mensaje, sizeof(mensaje),stdin) != NULL )	/*Se valida la entrada del menesaje*/			
write (tuberia[1], mensaje, strlen(mensaje) + 1);	/*Se muestra el mensaje ingresado previamente*/		
}
while( strcmp (mensaje,"FIN\n") != 0 );			/*Valida la salida del programa*/
close(tuberia[0]);						/*Cierra la tuberia del proceso padre*/
close(tuberia[1]);						/*Cierra la tuberia del proceso hijo*/
return (0);			
break;	
}
}
return (0);
}						/*FIN*/