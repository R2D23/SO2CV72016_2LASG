\section{Linux Shell}

	Responda lo siguiente:

		\begin{enumerate}

			\item ¿Quién y en qué año desarrolló el primer shell para UNIX?
			\item ¿Cuál es el nombre del primer shell para UNIX?
			
			\item Mencione tres variantes de shell diferentes a sh
			\item ¿Cuál es el shell en las distribuciones Linux?
			
			\item ¿Cuál es la extención de un Shell Script?
			
			\item Defina: pathname, relative path y absolute path
			
		\end{enumerate}

	Escriba la funci\'on de los siguientes comandos o utilizades: 
			
			\begin{itemize}

				\item man:
				\item cd:
				\item cp:
				\item mv: 
				\item ls: 
				\item rm: 
				\item pwd:
				\item rmdir: 
				\item mkdir:
				\item chmod: 
				\item chown:
				\item touch:
				\item less:
				\item cat:
				\item sudo:
				\item su:
				\item apt-get:
				\item clear:
				\item wget:
				\item who:
				\item whoami:
				\item passwd:
				\item date:
				\item uname:
				\item history:
				\item tar:
				\item shutdown:
				\item make:	
				\item printf: 
				\item finger:
				\item users:
				\item w:
				\item alias:

			\end{itemize}

	Escriba en el editor nano/pico el siguiente shell script (Obtenga pantalla):
	
	\begin{flushleft}
		$\sharp$ hello.sh \\
		$\sharp$ This is my first shell script \\
		printf “Hello! Bash is wonderful.” \\
		exit 0
	\end{flushleft}

	Guarde el archivo como hello.sh, cambie los modos de acceso para que el dueño y el grupo tengan todos los permisos y los demás ninguno (Obtenga pantalla). A continuación ejecute el shell script con bash y con (Obtenga pantalla de ambas ejecuciones)./