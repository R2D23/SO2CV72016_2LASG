\section{Linux}

Responda lo siguiente

	\begin{enumerate}

			\item ¿Qué es UNIX? 
			Es un Sistema Operativo desarrollado en los Laboratorios Bell, diseñado para Digital Equipment PDP computers, que posteriormente se volvió muy popular entre los usuarios, como un sistema operativo multitareas de amplia variedad en plataformas de hardware, desde computadoras personales hasta servidores de multiprosesadores y supercomputadoras.
			Éste Sistema Operativo se basa en ciertas especificaciones para su desarrollo, conocidas como especificaciones únicas de UNIX que definen el comportamiento de todas de las funciones de el Sistema Operativo UNIX, toas estas especificaciones son un conjunto de series de especificaciones que desarrolló la IEEE (Instituto de Ingenierías Eléctricas y Electrónicas, ó Institute of Electrical and Electronic Engineers por su idioma en inglés) las especificaciones P1003 o POSIX
			Los Sistemas Operativos de UNIX se basan en las características de: simplicidad, rendimiento, componentes reutilizables, filtros y archivos de formato abierto. 
			\item ¿Qué es y quién desarrolló Linux?
			Es un Sistema operativo de distribucion abierta, parecido al kernel de UNIX, núcleo de bajo nivel del sistema operativo, tomando como base a UNIX, Linux es muy similar, de manera que muchos de los programas y aplicaciones desarrollados para UNIX, pueden utilizarse de igual forma en Linux, sin siquiera modificarlo.
			Fue desarrollado por Linus Tolvalds de la Universidad de Helsinki, con la ayuda de algunos programadores de Unix a traves de la red, originalmente un hobby inspirado en el Minix de Andy Tanenbaum, un sistema con similitudes al UNIX. la intencion de Linus fue que su sistema no contuviera codigo rpopiedad de alguien, y por el contrario, que pudiera ser de libre distribucion.
			En la actualidad Linux tiene una amplia variedad de versiones con la posibilidad de ser usados en una gran cantidad de diferentes tipos de CPU's, incluyendo computadoras personalescon procesadores Intel x86 y procesadores compatibles.
			\item Describa qué es Open Source
			Es cualquier recurso que tiene permisos de libre distribucion, o que esta sujeto a las licencias de GNU de caracter público general
			\item ¿Qué es la Free Software Foundation?
			es una fundacion creada por Richard Stallman, el autor de los emacs de GNU y uno de los más conocidos editores de texto para UNIX y otros sistemas, Stallman es pionero en el concepto de software libre e inicio el proyecto de GNU, con el que intento crear un sistema operativo  y un ambiente que fuera compatible con UNIX, pero que no tuviera las restricciones de responder a propiedad de UNIX ni de su codigo fuente, siendo muy diferente éste al más bajo nivel, pero aún siendo compatible con las aplicaciones de UNIX
			\item ¿Qué se entiende por Software propietario?
			A aquel software que se tiene registrado como propiedad intelectual de alguna persona o institucion, y que no permite que se utilice alguno de sus componentes o codigo fuente que puedan usarse para crear otro software sin la autirizacion de su propietario.
			\item ¿Cuál es el significado de GNU?
			GNU No es Unix
			\item Describa en qué consiste el Copy-Left?
			Es tomado como una burla a Copyrights o derechos de autor, haciendo alusión a que sería un software totalmente libre de restricciones en su uso.
			\item De manera general, en qué consiste la licencia GNU-GPL?
			A la distribución gratuita de software, o a través de una Licencia Pública que evitan las restricciones de cualquier tipo.
			\item ¿Cúal es el nombre de la primera distribución comercial de Linux?
			La primera distribucion de Linux fue nombrada con el mismo nombre, debido a que en sí, el sistema operativo era Linux, como así mismo lo era el kernel o el nucleo del sistema.
	\end{enumerate}

